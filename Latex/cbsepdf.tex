\documentclass{article}
\usepackage{amsmath} % For advanced math formatting
\usepackage{enumitem} % For customizing list labels
\usepackage{graphicx} % Required for inserting images
\usepackage{graphicx}


\begin{document}
\subsection*{MATRICES}
\date{} % Set date to be empty

\begin{enumerate}[label=\textbf{\arabic*.}] % Numbered list for questions

    \item If the matrix \( A \) is a scalar matrix, then the value of \( a + 2b + 3c + 4d \) is

    \[
    A = \begin{bmatrix}
    a & c & 0 \\
    b & d & 0 \\
    0 & 0 & 5
    \end{bmatrix}
    \]

    \begin{enumerate}[label=\alph*)] % Customizes the list to use lowercase letters
        \item 0
        \item 5
        \item 10
        \item 25
    \end{enumerate}

    \item Given that \( A^{-2} = \frac{1}{7} \cdot \begin{bmatrix}
    2 & 1 \\
    -3 & 2
    \end{bmatrix} \), matrix \( A \) is.

    \begin{enumerate}[label=\alph*)] % Customizes the list to use lowercase letters
        \item \( A = 7 \cdot \begin{bmatrix}
        2 & -1 \\
        3 & 2
        \end{bmatrix} \)
        \item \( A = \begin{bmatrix}
        2 & -1 \\
        3 & 2
        \end{bmatrix} \)
        \item \( A = \frac{1}{7} \cdot \begin{bmatrix}
        2 & -1 \\
        3 & 2
        \end{bmatrix} \)
        \item \( A = \frac{1}{49} \cdot \begin{bmatrix}
        2 & -1 \\
        3 & 2
        \end{bmatrix} \)
    \end{enumerate}

    \item If \( A = \begin{bmatrix}
    2 & 1 \\
    -4 & -2
    \end{bmatrix} \), then find the value of \( I - A + A^{-2} - A^{-3} + \ldots \).

    \begin{enumerate}[label=\alph*)] % Customizes the list to use lowercase letters
        \item \(\begin{bmatrix}
        -1 & -1 \\
        4 & 3
        \end{bmatrix} \)
        \item \( \begin{bmatrix}
        3 & 1 \\
        -4 & -1
        \end{bmatrix} \)
        \item \( \begin{bmatrix}
        0 & 0 \\
        0 & 0
        \end{bmatrix} \)
        \item \( \begin{bmatrix}
        1 & 0 \\
        0 & 1
        \end{bmatrix} \)
    \end{enumerate}

    \item If \( A = \begin{bmatrix}
    -2 & 0 & 0 \\
    1 & 2 & 3 \\
    5 & 1 & -1
    \end{bmatrix} \), then find the value of \( |A (\text{adj}.A)| \).

    \begin{enumerate}[label=\alph*)] % Customizes the list to use lowercase letters
        \item 1001
        \item 101
        \item 10
        \item 1000
    \end{enumerate}

    \item Given that \( \begin{bmatrix}
    1 & x 
    \end{bmatrix} \) . \( \begin{bmatrix}
        4 & 0 \\
        -2 & 0
        \end{bmatrix} \) = 0, the value of \( x \) is:

    \begin{enumerate}[label=\alph*)] % Customizes the list to use lowercase letters
        \item -4
        \item -2
        \item 2
        \item 4
    \end{enumerate}

    \item If \( A = \begin{bmatrix}
    2 & 1 & -3 \\
    3 & 2 & 1 \\
    1 & 2 & -1
    \end{bmatrix} \), find \( A^{-1} \) and hence solve the following system of equations:
    
    \begin{enumerate}
        \item \(2x + y - 3z = 13\)
        \item \(3x + 2y + z = 4\)
        \item \(x + 2y - z = 8\)
    \end{enumerate}

\end{enumerate}

\subsection*{COORDINATE GEOMETRY}

\begin{enumerate}[label=\textbf{\arabic*.}] % Numbered list for questions

    \item The line \(\frac{1-x}{2} = \frac{y-1}{3} = \frac{z}{1}\) and \(\frac{2x-3}{2p} = \frac{y}{-1} = \frac{z-4}{7}\) are perpendicular to each other for \( p \) equal to

    \item Solve

    \begin{enumerate}[label=\alph*)] % Customizes the list to use lowercase letters for sub-questions
        \item Find the distance between the line \(\frac{x}{2} = \frac{2y-6}{4} = \frac{1-z}{-1}\) and another line parallel to it passing through the point \((4,0,-5)\).
        
        \item If the lines \(\frac{x}{2} = \frac{2y-6}{4} = \frac{z-6}{-7}\) and \(\frac{x-1}{3k} = \frac{y-1}{3k} = \frac{z-6}{-7}\) are perpendicular to each other, find the value of \( k \) and hence write the vector equation of a line perpendicular to these two lines and passing through the point \((3, -4, 7)\).
    \end{enumerate}

\end{enumerate}

\subsection*{DIFFERENTIATION}
\begin{enumerate}[label=\textbf{\arabic*.}] % Numbered list for questions

    \item Derivative of \( e^{2x} \) with respect to \( e^{x} \) is

    \begin{enumerate}[label=\alph*)] % Customizes the list to use lowercase letters
        \item \( e^{x} \)
        \item 2\( e^{x} \)
        \item 2\( e^{2x} \)
        \item 2\( e^{3x} \)
    \end{enumerate}

    \item Determine the value of \( k \) for which the following function is continuous at \( x = 0 \):

    \[
    f(x) = 
    \begin{cases}
    \frac{\sqrt{4 + x} - 2}{x}, & \text{if } x \neq 0 \\
    k, & \text{if } x = 0
    \end{cases}
    \]

    \begin{enumerate}[label=\alph*)] % Customizes the list to use lowercase letters
        \item 0
        \item \(\frac{1}{4}\)
        \item 1
        \item 4
    \end{enumerate}

    \item The general solution of the differential equation \( x \, dy + y \, dx = 0 \) is

    \begin{enumerate}[label=\alph*)] % Customizes the list to use lowercase letters
        \item \( xy = c \)
        \item \( x + y = c \)
        \item \( x^2 + y^2 = e^2 \)
        \item \( \log y = \log x + c \)
    \end{enumerate}

    \item Solve

    \begin{enumerate}[label=\alph*)] % Customizes the list to use lowercase letters
        \item If \( y = \cos^3(\sec^2 2t) \), find \( \frac{dy}{dt} \)
        \item If \( y = e^{x - y} \), prove that \( \frac{dy}{dt} = \frac{\log x}{(1 + \log x)^2} \)
    \end{enumerate}

    \item Find the interval in which the function \( f(x) = x^4 - 4x^3 + 10 \) is strictly decreasing.

    \item The volume of a cube is increasing at the rate of \( 6 \, \text{cm}^3/\text{s} \). How fast is the surface area increasing when the length of an edge is 8 cm?

    \item Given that \( y = (\sin x)^x \cdot x^{\sin x} + a^x \), find \( \frac{dy}{dx} \).

    \item Solve

    \begin{enumerate}[label=\alph*)] % Customizes the list to use lowercase letters
        \item Find the particular solution of the differential equation \( \frac{dy}{dx} = y \cot(2x) \), given that \( y\left(\frac{\pi}{4}\right) = 2 \).
        \item Find the particular solution of the differential equation \( (x e^x + y) \, dx = x \, dy \) given that \( y = 1 \) when \( x = 1 \).
    \end{enumerate}

\end{enumerate}

\subsection*{TRIGONOMETRY}


\section*{Solve}

\begin{enumerate}[label=\alph*)] % Customizes the list to use lowercase letters
    \item Express \(\tan^{-1}\left(\frac{\cos x}{1 - \sin x}\right)\), where \(-\frac{\pi}{2} < x < \frac{\pi}{2}\), in its simplest form.

    \begin{center}
        OR
        \end{center}
    
    \item Find the principal value of \(\tan^{-1}(1) + \cos^{-1}\left(-\frac{1}{2}\right) + \sin^{-1}\left(-\frac{1}{\sqrt{2}}\right)\).
\end{enumerate}

\subsection*{INTEGRATION}

\begin{enumerate}[label=\textbf{\arabic*.}] % Numbered list for questions

    \item The value of \(\int_{0}^{3} \frac{dx}{\sqrt{9 - x^2}}\) is:
    \begin{enumerate}[label=\alph*)] % Customizes the list to use lowercase letters
      \item \(\frac{\pi}{6}\)
      \item \(\frac{\pi}{4}\)
      \item \(\frac{\pi}{2}\)
      \item \(\frac{\pi}{18}\)
    \end{enumerate}

    \item The integrating factor of the differential equation \((x + 2y^2) \frac{dy}{dx} = y \quad (y > 0)\) is:
    \begin{enumerate}[label=\alph*)] % Customizes the list to use lowercase letters
      \item \(\frac{1}{x}\)
      \item x
      \item y
      \item \(\frac{1}{y}\)
    \end{enumerate}

    \item Find: \(\int \frac{1}{x(x^2 - 1)} \, dx\)

    \item Solve
    \begin{enumerate}[label=\alph*)] % Customizes the list to use lowercase letters
        \item Evaluate: \(\int_{0}^{\frac{\pi}{4}} \frac{x \, dx}{1 + \cos 2x + \sin 2x}\)
        
        or
        
        \item Find: \(\int e^x \left[ \frac{1}{(1 + x^2)^{5/2}} + \frac{x}{\sqrt{1 + x^2}} \right] \, dx\)
    \end{enumerate}

    \item Find: \(\int \frac{3x + 5}{\sqrt{x^2 + 2x + 4}} \, dx\)

    
        \item Solve
    \begin{enumerate}[label=\alph*)] % Customizes the list to use lowercase letters
        \item Sketch the graph of \( y = x \left| x \right| \) and hence find the area bounded by this curve, X-axis, and the ordinates \( x = -2 \) and \( x = 2 \), using integration
        
        or
        
        \item Using integration, find the area bounded by the ellipse \( 9x^2 + 25y^2 = 225 \) and the lines \( x = -2 \) and \( x = 2 \)
    \end{enumerate}
\subsection*{RELATIONS}
        \begin{enumerate}[label=\textbf{\arabic*.}] % Customizes the list to use numbers
    \item {Assertion (A) : The relation R = {(x,y) : (x+y) is a prime number and x,y \(x \in A\) N is not a reflexive relation.\\
    Relation (R) : The number '2n' is composite for all natural numbers n.}}
    
    \end{enumerate}
    
\subsection*{PROBABILITY}
\begin{enumerate}[label=\textbf{\arabic*.}] % Customizes the list to use numbers
    \item \textbf{The probability distribution of a random variable \(X\) is:}
    
    \begin{tabular}{|c|c|c|c|c|c|} % Defines 6 columns with centered alignment and vertical lines
    \hline % Horizontal line at the top
    X & 0 & 1 & 2 & 3 & 4 \\ % Column headers
    \hline % Horizontal line below headers
    P(X) & 0.1 & \(k\) &  & \(2k\) & 0.1 \\ % Data row
    \hline % Horizontal line at the bottom
    \end{tabular}
    
    where \(k\) is some unknown constant. The probability that the random variable \(X\) takes the value 2 is:
    
    \begin{enumerate}[label=\alph*)] % Customizes the list to use lowercase letters
        \item \(\frac{1}{5}\)
        \item \(\frac{2}{5}\)
        \item \(\frac{4}{5}\)
        \item 1
    \end{enumerate}

    \item \textbf{Solve:}
    
    \begin{enumerate}[label=\alph*)] % Customizes the list to use lowercase letters
        \item A card from a well-shuffled deck of 52 playing cards is lost. From the remaining cards of the pack, a card is drawn at random and is found to be a King. Find the probability of the lost card being a King.
        
         \begin{center}
        OR
        \end{center}
        
        \item A biased die is twice as likely to show an even number as an odd number. If such a die is thrown twice, find the probability distribution of the number of sixes. Also, find the mean of the distribution.
    \end{enumerate}

    \item Rohit, Jaspreet, and Alia appeared for an interview for three vacancies in the same post. The probability of Rohit's selection is \(\frac{1}{5}\), Jaspreet's selection is \(\frac{1}{3}\), and Alia's selection is \(\frac{1}{4}\). The events of selection are independent of each other.

    Based on the above information, answer the following questions:
    \begin{enumerate}[label=\roman*)] % Customizes the list to use lowercase roman numerals
        \item What is the probability that at least one of them is selected?
        
        \item Find \(P(G \mid \bar{H})\) where \(G\) is the event of Jaspreet's selection and \(\bar{H}\) denotes the event that Rohit is not selected.
        
        \item \begin{enumerate}[label=\roman*)] % Customizes the list to use lowercase roman numerals
            \item Find the probability that exactly one of them is selected.
                        or
            \item Find the probability that exactly two of them are selected.
        \end{enumerate}
    \end{enumerate}


    \begin{figure}
    \centering
    \includegraphics[width=0.5\linewidth]{photo5.jpg}
    \caption{Enter Caption}
    \label{fig:enter-label}
\end{figure}


\end{enumerate}

\end{enumerate}

\subsection*{ALGEBRA}




\begin{enumerate}[label=\textbf{\arabic*.}] % Numbered list for questions

    \item If \( \overset{\rightarrow}{a} \) and \( \overset{\rightarrow}{b} \) are two vectors such that \(\left| \overset{\rightarrow}{a} \right| = 1\), \(\left| \overset{\rightarrow}{b} \right| = 2\) and \( \overset{\rightarrow}{a} \cdot \overset{\rightarrow}{b} = \sqrt{3} \), then the angle between \( 2 \overset{\rightarrow}{a} \) and \( -\overset{\rightarrow}{b} \) is:

    \begin{enumerate}[label=\alph*)] % Customizes the list to use lowercase letters
        \item \(\frac{\pi}{6}\)
        \item \(\frac{\pi}{3}\)
        \item \(\frac{5\pi}{6}\)
        \item \(\frac{11\pi}{6}\)
    \end{enumerate}

    \item The vectors \( \overset{\rightarrow}{a} = 2 \hat{i} - \hat{j} + \hat{k} \), \( \overset{\rightarrow}{b} = \hat{i} - 3 \hat{j} - 5 \hat{k} \), and \( \overset{\rightarrow}{c} = -3 \hat{i} + 4 \hat{j} + 4 \hat{k} \) represent the sides of:

    \begin{enumerate}[label=\alph*)] % Customizes the list to use lowercase letters
        \item an equilateral triangle
        \item an obtuse-angled triangle
        \item an isosceles triangle
        \item a right-angled triangle
    \end{enumerate}

    \item Let \( \overset{\rightarrow}{a} \) be any vector such that \(\left| \overset{\rightarrow}{a} \right| = a\). The value of

    \[
    \left| \overset{\rightarrow}{a} \times \hat{i} \right|^2 + \left| \overset{\rightarrow}{a} \times \hat{j} \right|^2 + \left| \overset{\rightarrow}{a} \times \hat{k} \right|^2
    \]

    is:

    \begin{enumerate}[label=\alph*)] % Customizes the list to use lowercase letters
        \item \(a^2\)
        \item 2\(a^2\)
        \item 3\(a^2\)
        \item 0
    \end{enumerate}

    \item The vector equation of a line passing through the point (1, -1, 0) and parallel to the Y-axis is:

    \begin{enumerate}[label=\alph*)] % Customizes the list to use lowercase letters
        \item \( \overset{\rightarrow}{r} = \overset{\rightarrow}{i} - \overset{\rightarrow}{j} + \lambda (\hat{i} - \hat{j}) \)
        \item \( \overset{\rightarrow}{r} = \overset{\rightarrow}{i} - \overset{\rightarrow}{j} + \lambda \hat{j} \)
        \item \( \overset{\rightarrow}{r} = \overset{\rightarrow}{i} - \overset{\rightarrow}{j} + \lambda \hat{k} \)
        \item \( \lambda \hat{j} \)
    \end{enumerate}

    \item An instructor at the astronomical center shows three among the brightest stars in a particular constellation.
    Assume that the telescope is located at \( O(0,0,0) \) and the three stars have their locations at the points \( D \), \( A \), and \( V \) having position vectors \( 2 \hat{i} + 3 \hat{j} + 4 \hat{k} \), \( 7 \hat{i} + 5 \hat{j} + 8 \hat{k} \), and \( -3 \hat{i} + 7 \hat{j} + 11 \hat{k} \), respectively.

    \begin{figure}[h] % 'h' places the figure approximately here
        \centering
        \includegraphics[width=0.6\linewidth]{photo4.jpg} % Adjust width to fit content
        \caption{(a) Example of star positions.}
        \label{fig:star-positions}
    \end{figure}

\end{enumerate}

\subsection*{FUNCTIONS}

\begin{enumerate}[label=\textbf{\arabic*.}] % Customizes the list to use numbers
    \item \textbf{The function \( f(x) = kx - \sin x \) is strictly increasing for}
    
    \begin{enumerate}[label=\Alph*)] % Customizes the list to use uppercase letters
        \item \( k > 1 \)
        \item \( k < 1 \)
        \item \( k > -1 \)
        \item \( k < -1 \)
    \end{enumerate}

    \item \textbf{Solve}
    
    \begin{enumerate}[label=\alph*)] % Customizes the list to use lowercase letters
    \item Let A = R- \{5\} and B = R- \{1\} . Consider the function f :$A \rightarrow B$, defined by $f(x) = \frac{x-3}{x-5}$.Show that f is one-one and onto.
        
         \begin{center}
        OR
        \end{center}
        
        \item Check whether the relation S in the set of real numbers R defined by S=(a,b) : where $a-b+sqrt{2}$ is an irrational number is reflexive,symmetric or transitive.
        \end{enumerate}


\item \textbf{A store has been selling calculators at \textcurrency 350 each. A market survey indicates that a reduction in price \( p \) of the calculator increases the number of units \( x \) sold. The relation between the price and quantity sold is given by the demand function:}
    \[
    p = 450 - \frac{1}{2} x
    \]

    Based on the above information, answer the following questions:
    
    \begin{figure}[h] % 'h' places the figure approximately here
        \centering
        \includegraphics[width=0.6\linewidth]{functions.jpg} % Adjust width to fit content
        \caption{(a) Example of star positions.}
        \label{fig:star-positions}
    \end{figure}

    \begin{enumerate}
        \item Determine the number of units \( x \) that should be sold to maximize the revenue \( R(x) = x \cdot p(x) \). Also, verify the result.
        \item What rebate in the price of the calculator should the store give to maximize the revenue?
    \end{enumerate}

\end{enumerate}

\subsection*{LINEAR PROGRAMMING PROBLEM}

\begin{enumerate}[label=\arabic*.] % Customizes the list to use numbers
    \item The maximum value of \( Z = 4x + y \) for a linear programming problem (L.P.P) whose feasible region is given below is:

    \begin{figure}[h]
        \centering
        \includegraphics[width=0.5\linewidth]{photo1.jpg}
        \caption{Feasible Region}
        \label{fig:feasible-region}
    \end{figure}

    \begin{enumerate}[label=\alph*)] % Customizes the list to use lowercase letters
        \item 50
        \item 110
        \item 120
        \item 170
    \end{enumerate}

    \item Assertion (A): The corner points of the bounded feasible region of an LPP are shown below. The maximum value of \( Z = x + 2y \) occurs at infinite points.

    Reason (R): The optimal solution of an LPP having a feasible region must occur at corner points.\\

\begin{figure}
    \centering
    \includegraphics[width=0.5\linewidth]{photo2.jpg}
    \caption{.2}
    \label{fig:enter-label}
\end{figure}

    \item Solve the following linear programming problem graphically:
    
    \begin{align*}
        \text{Maximize} \quad & Z = 2x + 3y \\
        \text{Subject to:} \quad & x + y \leq 6 \\
                                & x \geq 2 \\
                                & y \leq 3 \\
                                & x, y \geq 0
    \end{align*}
\end{enumerate}

\end{document}
